\documentclass{article}
\usepackage[utf8]{inputenc}
\usepackage{multicol}
\usepackage{graphicx}
\usepackage{geometry}
\usepackage{color}
\usepackage{amssymb}
\usepackage{graphicx}
\usepackage{caption}
\usepackage{subcaption}


\begin{document}


\section{The Wind Farm Layout Optimization Problem}


\textit{Finding the number of wind turbines, and position for each wind turbine relative to each other to maximize the expected power production within a given budget.}\\


\noindent Installing wind turbines in wind farms, reduces both installation and maintenance costs. In spite of this advantage, installing wind turbines in a farm also has its disadvantages. When wind turbines are placed together, the speed that hits a turbine will be reduced by a wake of turbulence generated by turbines that are in front of it. This effect is called the wake effect, and will be discussed more thoroughly later. If not considered, the wake effect can lead to extensive power loss. Therefore, it is crucial to find a positioning of the turbines that reduces the wake effect. The wind farm layout optimization problem is the problem of finding the optimal number of turbines to be positioned, and the optimal position for the turbines relative to each other. The goal is to find a layout the produces a maximal amount of expected power within a given budget.


\end{document}