\documentclass{article}
\usepackage[utf8]{inputenc}
\usepackage{multicol}
\usepackage{graphicx}
\usepackage{geometry}
\usepackage{color}
\usepackage{amssymb}
\usepackage{graphicx}
\usepackage{caption}
\usepackage{subcaption}


\begin{document}


\section{Wind Farm Layout Optimization using Genetic Algorithms}

% Mosetti et al, 1994
In 1994, Mosetti et al. successfully utilized the genetic algorithm on the wind farm layout optimization problem \textcolor{red}{[Mosetti et al., 1994]}. To model the wind decay Mosetti et al. used a model similar to the Jensen model \textcolor{red}{[Jensen, 1983]}. Mosetti et al. divided the wind farm terrain into a $10\times10$ quadratic grid so that a wind turbine could be installed in the middle of each cell. The optimization problem would then be to find which cells wind turbines should be installed in, in order to maximize power production and minimize cost. With this representation, an individual of the genetic search could be represented as a binary string of length 100, where each index represents a cell in the grid, so that a value of 1 means that an wind turbine is installed in the corresponding cell, and a value of zero means that there is no wind turbine in the corresponding cell. The genetic algorithm used was a simple, single-population genetic search where the fittest individuals where selected for reproduction using crossover and mutation. The crossover operation was performed at random locations with probability $0.6 < P_c < 0.9$ and mutation was performed with probability $0.01 < P_m < 0.1$. The fitness of the individuals was determined by the objective function

\begin{equation}
Objective = \frac{1}{P_{total}}w_1 + \frac{cost_{total}}{P_{total}}w_2
\label{Objective function Mosetti et al.}
\end{equation}

where $P_{total}$ is the total energy produced in one year, $cost_{total}$ is a function of the number of wind turbines installed, and $w_1$ and $w_2$ are weights. In the current study, $w_1$ was kept small so that the focus would be on lowest cost per energy produced. The model was tested on a single type of turbines in three different scenarios (a) constant wind direction and intensity, (b) constant wind intensity, but from a 360$^{\circ}$ variable direction, and (c) a realistic wind scenario. This work laid the foundation of numerous studies of wind farm optimization using genetic algorithms. The genetic algorithms have been optimized, and they now run on much more realistic wind scenarios, and cost models.\\


\noindent Huang improved the results of Mosetti et al. by using a distributed genetic algorithm \textcolor{red}{[Huang, 2007]}. Huang stated that using a distributed genetic algorithm will not only decrease the computation time by using parallelism, but also achieve higher fitness because more of the solution space will be explored. The distributed algorithm distributed the population of 600 individuals among 20 demes, using a ring-topology as shown in figure \ref{Ring Topology}.


\begin{figure}[h!]
\begin{center}
\includegraphics[scale=0.3]{"Ring Topology"}
\caption{Ring topology example with 6 demes.}
\label{Ring Topology}
\end{center}
\end{figure}


The simulation was run for 2500 generations with the migration strategy that 3.3\% of the individuals with highest fitness was selected as migrants, to replace the individuals with lowest fitness in the new population every 20th generation. \\


\noindent \c{S}i\c{s}bot et al. published a case study of wind turbine placement on a wind farm at the Island G{\"o}k\c{c}eada, at the north east of the Aegean Sea \textcolor{red}{[\c{S}i\c{s}bot et al., 2010]}. A distributed genetic algorithm was used, but, unlike Huang, the individuals were evaluated based on a multiple objective functions; one that measures the total cost (installation and operational), and one that measure total power production. \c{S}i\c{s}bot et al. argue that in an environment with changing demands, the use of a multi-objective function gives the decision-makers the opportunity to evaluate the different designs based on cost and power production separately, without ill-informed, randomly generated weights.

The selection process used is a controlled, elitist process, meaning that not only the fittest, but also some individuals that can spread diversity to the population are selected for reproduction. The genetic algorithm returns a set of Pareto optimal solutions; a set of solutions that are not dominated by any other solution in the set. Stated more formally, solutions \textbf{y} is said to dominate solution \textbf{x} if

\begin{equation}
\forall i: f_i(\textbf{x}) \leq f_i(\textbf{y}) \hspace{1mm} and \hspace{1mm} \exists j: f_j(\textbf{x}) < f_j(\textbf{y}) 
\label{Dominated solutions}
\end{equation}

where $f_i$ is objective function \textcolor{red}{[Murata et al., 2001]}. Other interesting features of this study is the introductions of constraints on the individuals and constraints on the cost, meaning that individuals with wind turbines outside the area of the island, and individuals with costs larger than the budget are removed from the population. Even thought constraints on individuals are not in accordance with the nature genetic algorithms, they can be necessary when the algorithm is applied to a real problem. Another feature introduced in this paper is rectangular cells. The argument behind this decision is that the safe distance between wind turbines is dependent of the direction of the turbine. The minimum distance between turbines in prevailing wind is 8D, while the minimum distance between turbines in the crosswind is 2D. In spite of this attempt to make the wind scenario more realistic, it is critiqued because it operates with a constant wind direction and constant speed, using the average wind direction and speed measured at the Island \textcolor{red}{[Samorani, 2013]}.

















%\begin{thebibliography}{100} % 100 is a random guess of the total number of
%
%\bibitem{mosetti} Mosetti G., Poloni C., and Diviacco B., ``Optimization of wind turbine positioning in large windfarms by means of genetic algorithm," \emph{Journal of Wind Engineering and Industrial Aerodynamics}, Vol. 51, pp. 105-116, 1994.
%
%\bibitem{huang} Huang, H. S., ''Distributed Genetic Algorithm for Optimization of Wind Farm Annual Profits,''\emph{International conference on Intelligent Systems Applications to Power Systems}, 2007.
%
%\bibitem{sisbot} \c{S}i\c{s}bot S., Turgut {\"O}., Tun\c{c} M., and \c{C}amdali {\"U}., ''Optimal positioning of wind turbines on G{\"o}k\c{c}eada using multi-objective genetic algorithm,'' \emph{Wind Energy}, Vol. 1, pp. 297-306, 2010.
%
%\end{thebibliography}


\end{document}