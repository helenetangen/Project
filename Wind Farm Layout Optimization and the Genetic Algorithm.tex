\documentclass{article}
\usepackage[utf8]{inputenc}
\usepackage{multicol}
\usepackage{graphicx}
\usepackage{geometry}
\usepackage{color}
\usepackage{amssymb}
\usepackage{graphicx}
\usepackage{caption}
\usepackage{subcaption}


\begin{document}


\section{Wind Farm Layout Optimization using Genetic Algorithms}

% Mosetti et al, 1994
In 1994, Mosetti et al. successfully utilized the genetic algorithm on the wind farm layout optimization problem \textcolor{red}{[Mosetti et al., 1994]}. To model the wind decay Mosetti et al. used a model similar to the Jensen model \textcolor{red}{[Jensen, 1983]}.\\

\noindent Mosetti et al. divided the wind farm terrain into a $10\times10$ quadratic grid so that a wind turbine could be installed in the middle of each cell. The optimization problem would then be to find which cells wind turbines should be installed in, in order to maximize power production and minimize cost. With this representation, an individual of the genetic search could be represented as a binary string of length 100, where each index represents a cell in the grid, so that a value of 1 means that an wind turbine is installed in the corresponding cell, and a value of zero means that there is no wind turbine in the corresponding cell. The genetic algorithm used was a simple, single-population genetic search where the fittest individuals where selected for reproduction using crossover and mutation. The crossover operation was performed at random locations with probability $0.6 < P_c < 0.9$ and mutation was performed with probability $0.01 < P_m < 0.1$. The fitness of the individuals was determined by the objective function \\

\begin{equation}
Objective = \frac{1}{P_{total}}w_1 + \frac{cost_{total}}{P_{total}}w_2
\label{Objective function Mosetti et al.}
\end{equation}

\noindent where $P_{total}$ is the total energy produced in one year, $cost_{total}$ is a function of the number of wind turbines installed, and $w_1$ and $w_2$ are weights. In the current study, $w_1$ was kept small so that the focus would be on lowest cost per energy produced.\\

\noindent The model was tested on a single type of turbines in three different scenarios (a) constant wind direction and intensity, (b) constant wind intensity, but from a 360$^{\circ}$ variable direction, and (c) a realistic wind scenario. This work laid the foundation of numerous studies of wind farm optimization using genetic algorithms. The genetic algorithms have been optimized, and they now run on much more realistic wind scenarios, and cost models.\\


\end{document}