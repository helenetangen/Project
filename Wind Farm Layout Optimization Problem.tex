\documentclass{article}
\usepackage[utf8]{inputenc}
\usepackage{multicol}
\usepackage{graphicx}
\usepackage{geometry}
\usepackage{color}
\usepackage{amssymb}
\usepackage{graphicx}
\usepackage{caption}


\begin{document}

% The introduction is too simple. It's like its written by a five-year-old.
% Explain wake effect by ij, as Samorani.


\section{The wind farm layout optimization problem}


This section defines the wind farm layout optimization problem and explains the key factors that makes the problem so complex. 


\subsection{The wind farm layout optimization problem}


An overview of the wind farm layout optimization problem is presented by Samorani \textcolor{red}{[Samorani, 2013]}. Grouping of wind turbines in a wind farm decreases installation and maintenance cost. However, positioning of wind turbines in a farm also introduces new challenges. The power produced by wind turbines is largely dependent on the wind speed, therefore it is important that the wind speed that hits a wind turbine is as large as possible. The main challenge for wind farms is that a wind turbine positioned in front of another wind turbine will cause a wake of turbulence, meaning that the wind speed that hits the second wind turbine will be decreased. This effect is called "wake effect", and will be explained in more detail later. Since the goal is to produce as much power as possible it is very important to position the wind turbines so that the wake effect is minimal. Samorani states the wind farm layout optimization problem like this


\begin{quote}
\textit{''The wind farm layout optimization problem consists of finding the turbine positioning (wind farm layout) that maximizes the expected power production.''}
\end{quote}


\noindent Samorani's definition can be extended to include a cost constraint and a search for the optimal number of turbines. The extended formulation would then be "...finding the number of turbines and turbine positioning (wind farm) that maximizes the expected power production within a given budget."


\subsection{The wake effect}


Different mathematical models, that vary in both complexity and quality, exists to models to the wake effect. However, most research within the field of wind farm layout optimization use the Jensen model \textcolor{red}{[Jensen, 1983]}. The Jensen model is preferred both for its proven ability to accurately model the wind speed reduction caused by the wake effect, but also because of its simplicity. The Jensen model is explained below.


\subsubsection{The Jensen model}


\begin{figure}[h!]
\begin{center}
\includegraphics[scale=0.5]{project/images/"Wake Effect"}
\caption{The wake effect \textcolor{red}{[Samorani, 2013]}}
\label{Wake effect}
\end{center}
\end{figure}


In figure \ref{Wake effect} the small black rectangle represents a wind turbine, and the blue area behind it illustrates the area that is affected by the wake created by the wind turbine. In the figure, the wind is blowing from left to right with uniform wind speed of $U_0$. As the wind hits the wind turbine it creates a wake of turbulence behind it so that the wind speed at distance $x$ behind the wind turbine is $U < U_0$. The area behind the wind turbine that is affected by the wake at distance $x$ has the radius $r_1 = \alpha x + r_r$ where $r_r$ is the rotor radius and $\alpha$ is the entrainment constant constant which decides how fast the wake expands


\begin{equation}
\alpha = \frac{0.5}{ln\frac{z}{z_0}}
\label{Alpha}
\end{equation}

where $z$ is the hub height of the wind turbine and $z_0$ is a surface roughness constant. The wind speed $U$ is therefore \textcolor{red}{Jean help!}


\begin{equation}
U = U_0\left(1 - \frac{2a}{1 + \alpha\left(\frac{x}{r_1}\right)}\right)
\label{Alpha}
\end{equation}


where $a$ is called the axial induction factor that determines how quickly the wake expands, and $C_T$ is the thrust coefficient, a measure of the proportion of wind captured when the wind hits the wind turbine.


\begin{equation}
a = 0.5 \left(1 - \sqrt{1 - C_T}\right)
\end{equation}


\end{document}