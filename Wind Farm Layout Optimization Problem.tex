\documentclass{article}
\usepackage[utf8]{inputenc}
\usepackage{multicol}
\usepackage{graphicx}
\usepackage{geometry}
\usepackage{color}
\usepackage{amssymb}
\usepackage{graphicx}
\usepackage{caption}


\begin{document}


\section{The wind farm layout optimization problem}


%Brief sentece about why this section is included.


\subsection{The wind farm layout optimization problem}


An overview of the wind farm layout optimization problem is presented by Samorani \textcolor{red}{[Samorani, 2013]}. Grouping of wind turbines in a wind farm decreases installation and maintenance cost. However, positioning of wind turbines in a farm also introduces new challenges. The power produced by wind turbines is largely dependent on the wind speed, therefore it is important that the wind speed that hits a wind turbine is as large as possible. The main challenge for wind farms is that a wind turbine positioned in front of another wind turbine will cause a wake of turbulence, meaning that the wind speed that hits the second wind turbine will be decreased. This effect is called "wake effect", and will be explained in more detail later. Since the goal is to produce as much power as possible it is very important to position the wind turbines so that the wake effect is minimal. Samorani states the wind farm layout optimization problem like this


\begin{quote}
\textit{''The wind farm layout optimization problem consists of finding the turbine positioning (wind farm layout) that maximizes the expected power production.''}
\end{quote}


\subsection{The wake effect}


Different mathematical models to explain the wake effect exist, however most 


\begin{figure}[h!]
\begin{center}
\includegraphics[scale=0.5]{project/images/"Wake Effect"}
\caption{The wake effect \textcolor{red}{[Samorani, 2013]}}
\label{Wake effect}
\end{center}
\end{figure}






\end{document}