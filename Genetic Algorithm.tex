\documentclass{article}
\usepackage[utf8]{inputenc}
\usepackage{multicol}
\usepackage{graphicx}
\usepackage{geometry}
\usepackage{color}
\usepackage{amssymb}


\begin{document}


\section{Genetic Algorithm (GA)}


% Reference Holland (inventor)
% Reference Darwin (evolution)
% Reference Goldberg (description)


% Bla bla to complex to solve analytically.
Genetic Algorithms are search algorithms inspired by the theory of evolution. They operate on a population of initially randomly generated individuals that evolve through natural selection and reproduction. 


An initial population of individual solutions is generated and the fitness of each individual is calculated based on a fitness function, which from now on will be called an objective function. Based on their objective function values the fittest individuals are selected for reproduction. By combining genes of the parent solutions and perform genetic operations such as mutation, a new pool of solutions is generated. Since the solutions of the newly generated population is produced by recombining the fittest solutions from the initial population the average fitness of the newly generated population is expected to be higher than the one of the initial population. This process continues until some stopping condition is reach, and by then the average and best fitness of the population should be pretty high.


\end{document}