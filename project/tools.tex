\chapter{Technical Description}\label{chapter:technical}


Since this thesis is part of the Genetic and Evolutionary Computation Conference 2015, the conference has provided all the contestants with an API available on a GitHub. API's for different programming languages are provided, but the Java API will be used in this thesis. The API includes an implementation of a simple genetic algorithm, an evaluation method and twenty different wind scenarios. Section \ref{section:api} give an explanation of the provided API, section \ref{section:testsimulation} display the results from a few test simulations (which will not be included in the final thesis), and section \ref{section:futurework} provides a discussion of future work.


\section{Application User Interface (API)}\label{section:api}
A class diagram of the most important variables and methods of the provided API is displayed in figure \ref{Class Diagram}. As can be seen, the diagram consist of six classes. The GA.java class which contains a simple genetic algorithm, i.e. the whole process displayed in figure \ref{GA}. Layout evaluation is taken care of by the evaluation classes. WindFarmLayoutEvaluator.java is an abstract class, implemented by KusiakEvaluator.java and CompetitionEvaluator.java, where the last one connects the evaluation cost function of the competition to the online server of the competition. Next, the class called WindScenario.java is used to initialized one of the twenty wind scenarios provided by the competition. Running the program is simple, it only requires initializing the wind scenario, layout evaluator, and the genetic algorithm, and then start the run() method in the GA.java class.


\begin{figure}[h!]
\begin{center}
\includegraphics[scale=0.3]{images/"Class Diagram"}
\caption{Class diagram of the most important variables and methods from the API provided by GECCO 2015.}
\label{Class Diagram}
\end{center}
\end{figure}


\subsection{Genetic Algorithm}
The genetic algorithm provided is a simple genetic algorithm that uses a binary representation. Even though a binary representation is used, it is slightly different from the approaches listed in related work, because positions that are not allowed is removed from the representation. This is shown in figure \ref{Binary Representation (API)}. Two of the nine positions, highlighted with the color red, contains obstacles and are therefore not allowed, these are therefore removed from the individual. This means that even though the wind farm is partitioned into 9 squares, the individual is only represented with a binary string of size seven, since there is only seven legal positions. The selection-, crossover- and mutation methods are the ones presented in the section \ref{subsection:sga}.


\begin{figure}[h!]
\begin{center}
\includegraphics[scale=0.3]{images/"Binary Representation (API)"}
\caption{Binary representation of a wind farm. Red cells contains obstacles and are therefore not included in the binary representation. White cells represents empty positions, grey cells contains turbines.}
\label{Binary Representation (API)}
\end{center}
\end{figure}


\subsection{Fitness function}\label{subsection:fitnessfunction}


The main task of the evaluation classes is to calculate the fitness of each individual based on the fitness function.  The fitness function to be optimized is provided by GECCO, and is displayed in equation \ref{Objective function}.\\

\begin{small}
\begin{equation}
fitness =  \frac{ \left( c_t \cdot n + c_s \cdot \floor*{\frac{n}{m}} \right) \left( \frac{2}{3} + \frac{1}{3} \cdot e^{-0.00174n^2} \right) + c_{OM} \cdot n}{\left( \frac{1 - (1 + r)^{-y}}{r} \right) } \cdot \frac{1}{8760 \cdot P} + \frac{0.1}{n}
\label{Objective function} 
\end{equation}
\end{small}


\noindent Description and value of all parameters given in equation \ref{Objective function} are displayed in table \ref{Parameters}. As can be seen in this table, the values of $n$, the number of turbines, and $P$, farm energy output, are not given. This is because the number of turbines, together with the turbine positions, are the parameters to be optimized by the genetic algorithm. Farm energy output is the indirect parameter that we are trying to optimize, it is dependent on turbines count, position, wind scenario and so on, and is off course therefore not provided in table \ref{Parameters} either.\\


\begin{table}[h!]
\begin{center}
\caption{Description and value of each parameter used in the objective function provided by GECCO 2015.}
\label{Parameters}
\begin{tabular}{l|l|l}
\textbf{Parameter} & \textbf{Description} & \textbf{Value} \\ 
\hline 
$c_t$ & Turbine cost (usd) & 750 000 \\ 
$c_s$ & Substation cost (usd) & 8 000 000 \\ 
$m$ & Turbines per substation & 30 \\ 
$r$ & Interest rate & 0.03 \\ 
$y$ & Farm lifetime (years) & 20 \\ 
$c_{OM}$ & Yearly operating costs per turbine & 20 000 \\ 
$n$ & Number of turbines &  \\ 
$m$ & Farm energy output &  \\  
\end{tabular} 
\end{center}
\end{table}


\noindent Intuitively, the objective function can be divided into different parts. The first parenthesis in the nominator of the first fraction is the construction cost, while the second parenthesis is the economies of scale and the third part of the nominator is yearly operating costs. The denominator represents the interests. The denominator of the second fraction describes yearly power output, while the number $0.1$ in the nominator of the last fraction is a farm size coefficient. \\



\subsection{Wake-, Wind- and Power Model}\label{subsection:wakewindpower}
The evaluation class uses the same wake-, wind- and power model as \cite{Kusiak}. The wake model used is the classical Jensen model \citep{Jensen}, which is used in almost every study of the wind farm layout optimization problem, as can be seen in table \ref{table:overview}. \\

\noindent Wind distribution is modeled using the Weibull distribution, a continuous probability distribution shown to model wind distribution quite well \citep{Justus}. The probability density function is shown in equation \ref{equation:Weibull}


\begin{equation}
f(x; c, k)  = 
\begin{cases}
\frac{k}{c} \left( \frac{x}{c} \right)^{k-1} e^{- \left( \frac{x}{c} \right)^k } & \text{if} \hspace{1mm} x \geq 0 \\
0                                                                                                                      & \text{if} \hspace{1mm}     x < 0
\end{cases}
\label{equation:Weibull}
\end{equation}


\noindent where $k$ is called the shape parameter and $c$ is the scale parameter, and $k, c > 0$. In most of the wind scenarios provided by GECCO 2015, $k \approx 2$, this is shown empirically to be a good value for wind speed distribution \citep{Justus}. On the other hand, the shape parameter vary for each wind direction. Figure \ref{figure:weibull distribution} shows the Weibull distribution plotted for $k = 2$ and for different values of $c$. \\


\begin{figure}[h!]
\begin{center}
\includegraphics[scale=0.3]{images/"Weibull"}
\caption{The Weibull distribution plotted for $k = 2$ for different values of the scale parameter $c$.}
\label{figure:weibull distribution}
\end{center}
\end{figure}

\noindent The wind scenarios used in this thesis are therefore a specification of the shape- and scale parameters for every wind direction, where wind direction is partitioned into 24 different directions. Twenty wind scenarios are provided be GECCO 2015, ten which simply specify wind distribution parameters, and ten that specify wind distribution parameters and locations of obstacles. \\

\noindent The power curve used is also the same as used in \cite{Kusiak}, it is a linear function, shown in equation \ref{equation:Power Curv (API)},

\begin{equation}
 f(v) = 
  \begin{cases} 
   0                                  & \text{if }     v < v_{cut-in} \\
   \lambda v + \eta           & \text{if }     v_{cut-in} \leq v \leq v_{rated} \\
   P_{rated}                        & \text{if }     v_{cut-out} > v > v_{rated}. \\
  \end{cases}
  \label{equation:Power Curv (API)}
\end{equation}

\noindent Here $\lambda$ is the slope parameter, $v$ the wind speed, $\eta$ the intercept parameter, $P_{rated}$ is the fixed power output, and $v_{cut-in}$ is the cut-in speed; the minimum speed for which the turbine produces power, and $v_{cut-out}$ is the cut-out speed; the maximum wind speed for which the turbine is kept on. 


\section{Test Simulation}\label{section:testsimulation}
In order to demonstrate that the provided environment works, two test runs were performed. Figure \ref{Test Plots} shows the resulting fitness plots when the provided genetic algorithm was ran once for scenario 1 without obstacles, and once for scenario 1 with obstacles, using default parameters. Each simulation was run for 50 generations, ten times, and the plots show the average fitness over these ten runs. As can be seen, the fitness decreases as the population becomes better and better solutions to the problem. The scenario without obstacles are simpler and converges therefore to a lower fitness value than the scenario with obstacles. Note that these simulations are only ran to show that the simulation works, and the results will not be included in the final thesis. 


\begin{figure}[h!]
    \centering
    \begin{subfigure}[b]{0.45\textwidth}
        \includegraphics[width=\textwidth]{plots/"Fitness Scenario 1"}
        \caption{}
    \end{subfigure}
    \begin{subfigure}[b]{0.45\textwidth}
        \includegraphics[width=\textwidth]{plots/"Fitness Scenario 1 Obstacles"}
        \caption{}
    \end{subfigure}
    \caption{Minimum-, maximum-, and average fitness for two scenarios provided by GECCO; (a) Fitness for scenario 1, without obstacles, (b) Fitness for scenario 1 with obstacles. Each simulation was run for 50 generations, ten times, and the result is averaged over these ten runs.}
    \label{Test Plots}
\end{figure}


\section{Future Work}\label{section:futurework}
In order to investigate the goal statement and answer the research questions the provided genetic algorithm needs to be extended as shown in figure \ref{Class Diagram DGA}. The provided genetic algorithm will be used as a super class, which will be extended by seven different distributed genetic algorithms; the master-slave model, the Island model, the cellular model, three different hybrid models and the pool model, all described in detail in section \ref{subsection:dga}. Even though the hybrid models are drawn as one class in \ref{Class Diagram DGA} they will be implemented as three different classes. \\

\begin{figure}[h!]
\begin{center}
\includegraphics[scale=0.2]{images/"Class Diagram DGA"}
\caption{Simplified version of class diagram from figure \ref{Class Diagram} extended with different distributed genetic algorithms (green).}
\label{Class Diagram DGA}
\end{center}
\end{figure}

\noindent After the simple genetic algorithm and the population distributed genetic algorithms are implemented, extensive simulations will be run for all the models. In order to be able to run simulations on a personal computer, each model, even the simple genetic algorithm, will use parallelism. Each model will be compared based on five parameters; (1) fitness value, (2) total power generated, (3) efficiency, (4) number of turbines in solution layout, and (5) total cost. Efficiency is a measure of how much energy, out of the maximum energy amount that would have been produced without wake effect, the final layout is able to produce. \\

\noindent In addition to implementing the distributed genetic algorithms listed above, there might be a few changes done to the class GA.java depending on the ''goodness'' of the results, such as changes to the selection-, mutation and/or crossover methods. Other extensions might also be included if needed, such as seeding the genetic algorithm in order to get as good results as possible for the competition.\\

