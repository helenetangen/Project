\chapter{Introduction}
This thesis is a contribution to a contest launched by the annual international conference of evolutionary computation (GECCO) 2015, involving optimizing the number of turbines, and turbine positions in a wind farm with goal of producing maximum power to minimum cost. A wind farm simulator is provided by GECCO 2015, and therefor will the focus of this thesis be on improving the genetic algorithm that will be used to search for the optimal turbine positioning.


\section{Motivation and Background}
Transitioning from non-renewable energy sources to renewable energy sources is one of the largest, if not the larges political challenge of today. Renewable energy is less polluting that non-renewable energy and should therefore be preferred. However, renewable energy sources make up only 17.05 percent of the worlds energy sources as of 26th of May 2015 (renewableenergyworld.com). Wind turbine technology is a promising source of renewable energy. Wind turbine technology advances has led to wind turbines able to produce more energy to lower costs. However, wind turbines still produces less energy than predicted because of the wake effect; reduction in wind speed caused by turbines placed in front of other turbines \citep{Samorani}. For wind energy to become a bigger player in the worlds energy sources, sophisticated methods for wind turbine placement in wind farms needs to be developed so that each turbine produces as much energy as possible. \\

\noindent Wind turbine positioning is hard to optimize analytically. Fortunately, a wide variety of local search and bio-inspired methods have shown promising results, with genetic algorithms being the most popular method. As more advanced approaches to evaluate layouts are developed, and more realistic constrains are introduced, more sophisticated genetic algorithms are required. To come up with more advanced genetic algorithms for solving the wind farm layout optimization problem, the annual Genetic and Evolutionary Computation Conference (GECCO), launched a competition where different contestants provide their own implementation of a genetic algorithm. The goal of the competition is to bring more realistic problems to algorithm developers, and to create an open source library useful beyond the scope of the competition (http://www.irit.fr/wind-competition/). Wind parameters and evaluation mechanism are provided by GECCO, therefore the goal of this project will be to optimize the genetic algorithm for solving the problem not wind farm parameters and models. However, knowledge of wind turbines, wind farm layout and wake models are useful in understanding the project and is therefore introduced in the background section. \\

\noindent Greedy heuristics, Monte Carlo simulation, simulated annealing search, ant colony algorithms, particle swarm optimization and genetic algorithms have all been used in solving the wind farm layout optimization problem, and these will all be reviewed in chapter \ref{chapter:relatedwork}. Turbine positioning have been improved by genetic algorithms for more than 20 years, each approach bringing something new to the field such as a new type of genetic algorithm, a more realistic environment, and genetic algorithms combined with other search approaches. Many authors have implemented a population distributed genetic algorithms using the Island model, and it has shown promising results. However, as far as the author knows, no attempt has been made in implementing any of the other population distributed models. This fact is the main motivation behind this thesis, and have inspired the author to investigate the effect different population distributed algorithms can have on the wind farm layout optimization problem. 


\section{Goal and Research Questions}
This section states the goal statement and research questions that will be investigated in this thesis. \\

\noindent \textbf{Goal statement}

\begin{quote}
\textit{The project goal is to investigate the advantages of using distributed genetic algorithms to optimizing wind farm layout, i.e. solving the wind farm layout optimization problem.} \citep{Samorani}
\end{quote}

\noindent The performance of distributed genetic algorithms will be studied and compared to the performance of a simple genetic algorithm (not distributed) as well as to each other, with the goal of answering the research questions stated below.\\

\noindent \textbf{Research question 1}

\begin{quote}
\textit{Can distributed genetic algorithms improve the quality of the solution to the wind farm layout optimization problem as compared to simple genetic algorithm.}
\end{quote}

\noindent \textbf{Research question 2}

\begin{quote}
\textit{Which distributed genetic algorithm works best for the wind farm layout optimization problem? What properties are essential for its success?}
\end{quote}

\noindent Both research questions will be tested on a wind farm simulator provided by GECCO 2015, on twenty different wind scenarios which are also provided by the contest. Research question 1 will be answered by implementing different types of population distributed genetic algorithms and compare their results with a simple genetic algorithm on each wind scenario. Research question 2 will be tested the same way, and the results of each of the population distributed algorithms will be compared.


\section{Thesis Structure}
The thesis is divided into four chapters. Chapter 2 contains an introduction to the wind farm layout optimization problem, a description of how the genetic algorithm works, and a description of each of the distributed genetic algorithms that will be implemented. Chapter 3 is a survey of the state of the art within wind farm layout optimization, one section describing approaches using genetic algorithms, and one section describing other approaches. Chapter 4 is an description of the application user interface provided by GECCO 2015 that will be extended with different distributed genetic algorithm implementations, a test simulation, and a discussion of future work. 