\chapter{Related Work}


\section{Related Work}


% Mosetti et al, 1994
In 1994, Mosetti et al. successfully utilized the genetic algorithm on the wind farm layout optimization problem {\citep{Mosetti}}. To model the wind decay Mosetti et al. used a model similar to the Jensen model \textcolor{red}{[Jensen, 1983]}. Mosetti et al. divided the wind farm terrain into a $10\times10$ quadratic grid so that a wind turbine could be installed in the middle of each cell. The optimization problem would then be to find which cells wind turbines should be installed in, in order to maximize power production and minimize cost. With this representation, an individual of the genetic search could be represented as a binary string of length 100, where each index represents a cell in the grid, so that a value of 1 means that an wind turbine is installed in the corresponding cell, and a value of zero means that there is no wind turbine in the corresponding cell. The genetic algorithm used was a simple, single-population genetic search where the fittest individuals where selected for reproduction using crossover and mutation. The crossover operation was performed at random locations with probability $0.6 < P_c < 0.9$ and mutation was performed with probability $0.01 < P_m < 0.1$. The fitness of the individuals was determined by the objective function

\begin{equation}
Objective = \frac{1}{P_{total}}w_1 + \frac{cost_{total}}{P_{total}}w_2
\label{Objective function Mosetti et al.}
\end{equation}

where $P_{total}$ is the total energy produced in one year, $cost_{total}$ is a function of the number of wind turbines installed, and $w_1$ and $w_2$ are weights. In the current study, $w_1$ was kept small so that the focus would be on lowest cost per energy produced. \\

\noindent The model was tested on a single type of turbines in three different scenarios (a) constant wind direction and intensity, (b) constant wind intensity, but from a 360$^{\circ}$ variable direction, and (c) a realistic wind scenario. \textcolor{red}{Feil!} For each scenario, the results was measured against random configurations of 50 turbines. In scenario one, the efficiency of the random configuration was 0.50, while the efficiency of the optimal solution was 0.95. In the second wind scenario, the efficiency was decreased from 0.35 in the random configuration to 0.88 in the optimized configuration. And, in the last scenario the efficiency was increased from 0.34 to 0.84. In each case, the genetic algorithm was able to increase the total energy produced \textcolor{red}{Feil!} and reduce the cost \textcolor{red}{Feil!}. For each scenario the number of wind turbines was decreased drastically in the optimized version. Table \ref{Results Mosetti et al.} summarizes the results obtained.

\begin{center}
\begin{table}[h!]
\caption{Optimized configurations compared against random configurations for each of the three scenarios (a) single wind direction, (b) multiple wind direction with constant intensity and (c) multiple wind direction and intensity {\citep{Mosetti}}.}
\label{Results Mosetti et al.}
\scalebox{0.8}{
\begin{tabular}{c|c|c|c|c|c}
Scenario & Configuration & Efficiency & $P_{tot}$(kWyear) & cost/kWyear & Number of turbines \\ 
\hline 
(a) & Random    & 0.50 & 13025 & $2.57 \times 10^{-3}$ & 50 \\ 
      & Optimized & 0.95 & 12375 & $1.57 \times 10^{-3}$ & 25 \\ 
(b) & Random    & 0.35 & 9117   & $3.68 \times 10^{-3}$ & 50 \\ 
      & Optimized & 0.88 & 8711   & $1.84 \times 10^{-3}$ & 19 \\ 
(c) & Random    & 0.34 & 4767    & $7.04 \times 10^{-3}$ & 50 \\ 
      & Optimized & 0.84 & 3695    & $3.61 \times 10^{-3}$ & 15 \\ 
\end{tabular}}
\end{table}
\end{center}


This work laid the foundation for a number of studies of wind farm optimization using genetic algorithms. The genetic algorithms have been optimized, and they now run on much more realistic wind scenarios, cost-, and power models.\\