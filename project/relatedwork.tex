\chapter{Related Work}


\section{Related Work}


% Mosetti et al, 1994 --------------------------------------------------------------------------------------------------------------------------------------------------


In 1994, Mosetti et al. successfully utilized the genetic algorithm on the wind farm layout optimization problem {\citep{Mosetti}}. To model the wind decay Mosetti et al. used a model similar to the Jensen model \textcolor{red}{[Jensen, 1983]}. Mosetti et al. divided the wind farm terrain into a $10\times10$ quadratic grid so that a wind turbine could be installed in the middle of each cell. The optimization problem would then be to find which cells wind turbines should be installed in, in order to maximize power production and minimize cost. With this representation, an individual of the genetic search could be represented as a binary string of length 100, where each index represents a cell in the grid, so that a value of 1 means that an wind turbine is installed in the corresponding cell, and a value of zero means that there is no wind turbine in the corresponding cell. The genetic algorithm used was a simple, single-population genetic search where the fittest individuals where selected for reproduction using crossover and mutation. The crossover operation was performed at random locations with probability $0.6 < P_c < 0.9$ and mutation was performed with probability $0.01 < P_m < 0.1$. The fitness of the individuals was determined by the objective function

\begin{equation}
Objective = \frac{1}{P_{total}}w_1 + \frac{cost_{total}}{P_{total}}w_2
\label{Objective function Mosetti et al.}
\end{equation}

where $P_{total}$ is the total energy produced in one year, $cost_{total}$ is a function of the number of wind turbines installed, and $w_1$ and $w_2$ are weights. In the current study, $w_1$ was kept small so that the focus would be on lowest cost per energy produced. \\

\noindent The model was tested on a single type of turbines in three different scenarios (a) single wind direction, (b) multiple wind direction with constant intensity, and (c) multiple wind direction and intensity. For each scenario, the results was measured against random configurations of 50 turbines. In scenario one, the efficiency of the random configuration was 0.50, while the efficiency of the optimal solution was 0.95. In the second wind scenario, the efficiency was decreased from 0.35 in the random configuration to 0.88 in the optimized configuration. And, in the last scenario the efficiency was increased from 0.34 to 0.84. The genetic algorithm was able to reduce the total cost with approximately 50\% for each scenario, while the total energy production was decreased with 29\%. \textcolor{red}{Feil!} For each scenario the number of wind turbines was decreased drastically in the optimized version. Table \ref{Results Mosetti et al.} summarizes the results obtained.

\begin{center}
\begin{table}[h!]
\caption{Optimized configurations compared against random configurations for each of the three scenarios (a) single wind direction, (b) multiple wind direction with constant intensity and (c) multiple wind direction and intensity {\citep{Mosetti}}.}
\label{Results Mosetti et al.}
\scalebox{0.8}{
\begin{tabular}{c|c|c|c|c|c}
Scenario & Configuration & Efficiency & $P_{tot}$(kWyear) & cost/kWyear & Number of turbines \\ 
\hline 
(a) & Random    & 0.50 & 13025 & $2.57 \times 10^{-3}$ & 50 \\ 
      & Optimized & 0.95 & 12375 & $1.57 \times 10^{-3}$ & 25 \\ 
(b) & Random    & 0.35 & 9117   & $3.68 \times 10^{-3}$ & 50 \\ 
      & Optimized & 0.88 & 8711   & $1.84 \times 10^{-3}$ & 19 \\ 
(c) & Random    & 0.34 & 4767    & $7.04 \times 10^{-3}$ & 50 \\ 
      & Optimized & 0.84 & 3695    & $3.61 \times 10^{-3}$ & 15 \\ 
\end{tabular}}
\end{table}
\end{center}


As discussed in the paper, different simplifying assumptions are made in the model such as the wake effect model, the cost function, turbine type and layout model. The results are also only compared against random configurations, not configurations optimized by other optimization approaches, and no attempt is made to optimize the software. However, the purpose of this initial paper was to demonstrate the applicability of genetic algorithms on the wind farm layout optimization problem, and it laid the foundation for a number of studies of wind farm optimization using genetic algorithms. \\


% Grady et al. 2005 ----------------------------------------------------------------------------------------------------------------------------------------------------


\noindent In 2005, Grady et al. picked up where Mosetti et al. left of \textcolor{red}{[Grady et al., 2005]}. They recognized that while the results of Mosetti et al. beat random configurations they were not close to configurations made on simple empirical placement schemes. In their study, they opt to determine the effectiveness of the genetic algorithm compared to optimal configurations to investigate their usefulness in wind farm layout optimization. \\

\noindent As Mosetti et al., they also used a wake model similar to the Jensen model \textcolor{red}{(do I have to reference it again?)} as well as the same cost- and power function. However, the objective function was changed into

\begin{equation}
Objective = \frac{cost}{P_{tot}}
\label{Objective function Grady et al.}
\end{equation}

to minimize the cost per unit of produced energy, losing the weight from Mosetti et al presented in equation \ref{Objective function Mosetti et al.}.\\

\noindent The same three scenarios as Mosetti et al. was considered. However, the number of individuals was increased from 200 to 600, and run for 3000 generations instead of 400. Grady et al. also implemented a distributed genetic algorithm where the individuals was divided into 20 sub-populations, sadly not sharing any more implementations details. On the first scenario, Grady et al. recognized that with uniform wind distribution the optimal solution could be obtained by optimizing on single row of the layout, and copy it to the rest. As opposed to Mosetti et al., their results are identical to the optimal solution. In scenario (b) and (c) however, the optimal solution can not be obtained empirical, and therefore the results are just compared against those of Mosetti et al. The results for each scenario is displayed in table \ref{Results Grady et al.}. 


Table \ref{Results Grady et al.} compare the solutions of the two studies. The first thing to notice is the difference in number of turbines, where Grady et al. ends up with more turbines in each case, approximately doubling the number of turbines in scenario (b) and (c). The explanation behind this observation is the objective functions. Objective function \ref{Objective function Mosetti et al.} prioritizes low cost and hence prioritizes a lower turbine count. \textcolor{red}{Fact check!} The number of turbines for each case explains the largely explains the results. For each scenario the fitness of Mosetti et al. is higher than the fitness obtained in Grady et al. With exception of the first scenario, the efficiency is also larger in Mosetti et al., which makes sense since fewer turbines leads to less wake effect to decrease efficiency \textcolor{red}{Fact check!}. However, in each case, the total power production is largely increased in the current study, which also makes sense based on the turbine count \textcolor{red}{Fact check!}. 


\begin{table}[h!]
\begin{center}
\caption{Current results compared against the restults from Grady et al. for each of the three scenarios. \textcolor{red}{[Grady et al., 2005]}}
\label{Results Grady et al.}
\begin{tabular}{c|l|l|l}
Scenario & Parameter                    & Mosetti et al.  & Grady et al. \\ 
\hline 
(a)          & Fitness                        & 0.0016197     & 0.0015436 \\ 
              & Total power (kW year) & 12 352           & 14 310 \\ 
              & Efficiency (\%)            & 91.645            & 92.015 \\ 
              & Number of turbines     & 26                    & 30  \\ 
&&&\\
(b)         & Fitness                        & 0.0017371      & 0.0015666 \\ 
              & Total power (kW year) & 9244.7           & 17220 \\ 
              & Efficiency (\%)            & 93.859           & 85.174 \\ 
              & Number of turbines     & 19                   & 39 \\ 
&&&\\
(c)         & Fitness                        & 0.00099405   & 0.00080314 \\ 
              & Total power (kW year) & 13 460          & 32 038 \\ 
              & Efficiency (\%)            & 94.62            & 86.619 \\ 
              & Number of turbines     & 15                  & 39 \\  
\end{tabular} 
\end{center}
\end{table}


\noindent In summary, Grady et al. is able to show that an optimal solution for scenario (a) can be obtained by genetic algorithm search. It also shows that the power production obtained in Mosetti et al. can be increased by optimizing of parameter values and more sophisticated implementation of the genetic algorithm. However, they make no attempt to compare their solutions for scenario (b) and (c) to solutions obtained using other optimization techniques. 







