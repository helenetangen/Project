\chapter{Related Work}\label{chapter:relatedwork}
Wind farm layout optimization has been studied extensively over the last 20 years and the goal of this section is to provide the reader with an overview. This section is divided into three parts; Section \ref{section:relatedworkga} gives an extensive overview of wind farm layout optimization using the genetic algorithm, since genetic algorithms are the main focus of this thesis. Section \ref{section:relatedworkother} gives a short review of other optimization approaches, and section \ref{section:relatedworkdiscussion} contains a summary/discussion of related work.


\section{Wind Farm Layout Optimization using Genetic Algorithms}\label{section:relatedworkga}


% Mosetti et al, 1994.
\cite{Mosetti} were the first to successfully demonstrate the utilization of the genetic algorithm in solving the wind farm layout optimization problem. Although their work was made for illustrative purposes only, it laid the foundation for a number of more extensive studies of wind farm layout optimization using genetic algorithms.\\

\noindent In order to model a wind farm, a wake model, a power curve and a cost function needs to be specified. To model the wind decay, Mosetti et al. used a wake model similar to the one developed by \cite{Jensen}. Power generated by each turbine $i$ was modeled as a cubic function of the wind speed $u$ and site roughness $z_0$, and summed up to get the total power produced by the farm in one year as shown in equation \ref{Power Mosetti et al.}. Cost was modeled as a simple function of the number of turbines $N_t$, assuming the cost/year of a single turbine is $1$, and that a maximum cost reduction of $\frac{1}{3}$ can be obtained for each turbine if a large number of machines are installed, as shown in equation \ref{Cost Mosetti et al.} \\


\begin{equation}
\label{Power Mosetti et al.}
Power_{total} = \sum^{N_t}_{i} z_0u_i^3,
\end{equation}
%\todo{Do I have to explain where these come from?}


\begin{equation}
\label{Cost Mosetti et al.}
cost_{total} = N_t \left( \frac{2}{3} + \frac{1}{3}e^{-0.00174N_t^2} \right).
\end{equation}

\noindent With the goal of producing a great amount of power at low cost, the objective function to be minimized was formulated as a function of equation \ref{Power Mosetti et al.} and \ref{Cost Mosetti et al.}\\


\begin{equation}
Objective = \frac{1}{P_{total}}w_1 + \frac{cost_{total}}{P_{total}}w_2
\label{Objective function Mosetti et al.}
\end{equation}


\noindent where $w_1$ and $w_2$ are weights. In the current study, $w_1$ was kept small so that the focus would be on lowest cost per energy produced. \\

\noindent Mosetti et al. divided the wind farm terrain into a $10\times10$ quadratic grid where a wind turbine could be installed in the middle of each cell. The optimization problem would then be to position turbines in cells in a way that maximize power production and minimize cost. With this representation, an individual of the genetic search could be represented as a binary string of length 100, where each index represents a cell in the grid, so that a value of 1 means that a wind turbine is installed in the corresponding cell, and a value of zero means that there is no wind turbine in the corresponding cell. Figure \ref{Wind Farm Representation} illustrates how an individual represents a wind park for a wind farm partitioned into 100 cells. The genetic algorithm used was a simple, single-population genetic algorithm where the fittest individuals were selected for reproduction. The crossover operation was performed at random locations with probability $0.6 < P_c < 0.9$ and mutation was performed with probability $0.01 < P_m < 0.1$. \\


\begin{figure}[h!]
\begin{center}
\includegraphics[scale=0.3]{images/"Wind Farm Representation"}
\caption{An example of how the wind farm is represented in the genetic search from \cite{Mosetti}. An individual is represented as a bit-string of size 100, where each cell can either contain the value 1 or 0, representing a position containing a turbine and a position not containing a turbine, respectively.}
\label{Wind Farm Representation}
\end{center}
\end{figure}


\noindent The model was tested using a single turbine type on three different wind scenarios; (a) single wind direction, (b) multiple wind directions with constant intensity, and (c) multiple wind directions and intensities. For each scenario, the results were measured against random configurations of 50 turbines. In scenario (a), the efficiency of the random configuration was 0.50, while the efficiency of the optimized solution was 0.95. In (b), the efficiency was increased from 0.35 in the random configuration to 0.88 in the optimized configuration. And, in (c) the efficiency was increased from 0.34 to 0.84. For each scenario the number of wind turbines was decreased drastically in the optimized version. Table \ref{Results Mosetti et al.} summarizes the results obtained.

\begin{center}
\begin{table}[h!]
\caption{Optimized configurations compared against random configurations for each of the three scenarios (a) single wind direction, (b) multiple wind directions with constant intensity and (c) multiple wind directions and intensities{\citep{Mosetti}}.}
\label{Results Mosetti et al.}
\scalebox{0.8}{
\begin{tabular}{c|c|c|c|c|c}
Scenario & Configuration & Efficiency & $P_{tot}$(kWyear) & cost/kWyear & Number of turbines \\ 
\hline 
(a) & Random    & 0.50 & 13025 & $2.57 \times 10^{-3}$ & 50 \\ 
      & Optimized & 0.95 & 12375 & $1.57 \times 10^{-3}$ & 25 \\ 
(b) & Random    & 0.35 & 9117   & $3.68 \times 10^{-3}$ & 50 \\ 
      & Optimized & 0.88 & 8711   & $1.84 \times 10^{-3}$ & 19 \\ 
(c) & Random    & 0.34 & 4767    & $7.04 \times 10^{-3}$ & 50 \\ 
      & Optimized & 0.84 & 3695    & $3.61 \times 10^{-3}$ & 15 \\ 
\end{tabular}}
\end{table}
\end{center}


\noindent As discussed in the publication, different simplifying assumptions were made in the model, such as the cost function, only one turbine type and the layout model. The results are also only compared against random configurations, not configurations optimized by other optimization approaches, and no attempt has been made to optimize the software. However, the purpose of this initial paper was to demonstrate the applicability of genetic algorithms on the wind farm layout optimization problem, and it has certainly laid the ground work for a number of studies performed over the last 20 years. \\


% Grady et al. 2005.
\noindent \cite{Grady} picked up where \cite{Mosetti} left of. They recognized that while the results of Mosetti et al. beat random configurations they were not close to beat configurations made by simple empirical placement schemes. In their study, they wanted to show that by implementing a population distributed genetic algorithm the effectiveness of the algorithm could also be compared to optimal configurations.	 As Mosetti et al., they used the Jensen wake decay model, as well as the same cost- and power function, shown in equation \ref{Cost Mosetti et al.} and \ref{Power Mosetti et al.} respectively. However, the objective function was changed into the following

\begin{equation}
Objective = \frac{cost}{P_{tot}}.
\label{Objective function Grady et al.}
\end{equation}


\noindent The same three scenarios as Mosetti et al. were considered. However, the number of individuals was increased from 200 to 600, and run for 3000 generations instead of 400. The population distributed model used was an Island model where the individuals was divided into 20 sub-populations. Sadly, not many implementation details were shared. On the first scenario, Grady et al. recognized that with uniform wind distribution the optimal solution could be obtained by optimizing on single row of the layout, and copy it to the rest. As opposed to Mosetti et al., their results are identical to the optimal solution. In scenario (b) and (c) however, the optimal solution can not be obtained empirical, and therefore the results are just compared against those of Mosetti et al. The results for each scenario is displayed in table \ref{Results Grady et al.}. \\

\noindent Table \ref{Results Grady et al.} compare the solutions of the two studies. The first thing to notice is the difference in number of turbines, where Grady et al. ends up with more turbines in each case, approximately doubling the number of turbines in scenario (b) and (c). The explanation behind this observation is the objective function. Objective function \ref{Objective function Mosetti et al.} prioritizes low cost and hence prioritizes a lower turbine count. The number of turbines for each case largely explains the results. For each scenario the fitness of Mosetti et al. is higher than the fitness obtained by Grady et al. With exception of the first scenario, the efficiency is also larger in Mosetti et al., which makes sense since fewer turbines leads to less wake effect to decrease efficiency. However, in each case, the total power production is largely increased in the current study, which also makes sense based on the turbine count. \\


\begin{table}[h!]
\begin{center}
\caption{Current results compared against the results from Grady et al. for each of the three scenarios \citep{Grady}.}
\label{Results Grady et al.}
\begin{tabular}{c|l|l|l}
Scenario & Parameter                    & Mosetti et al.  & Grady et al. \\ 
\hline 
(a)          & Fitness                        & 0.0016197     & 0.0015436 \\ 
              & Total power (kW year) & 12 352           & 14 310 \\ 
              & Efficiency (\%)            & 91.645            & 92.015 \\ 
              & Number of turbines     & 26                    & 30  \\ 
&&&\\
(b)         & Fitness                        & 0.0017371      & 0.0015666 \\ 
              & Total power (kW year) & 9244.7           & 17220 \\ 
              & Efficiency (\%)            & 93.859           & 85.174 \\ 
              & Number of turbines     & 19                   & 39 \\ 
&&&\\
(c)         & Fitness                        & 0.00099405   & 0.00080314 \\ 
              & Total power (kW year) & 13 460          & 32 038 \\ 
              & Efficiency (\%)            & 94.62            & 86.619 \\ 
              & Number of turbines     & 15                  & 39 \\  
\end{tabular} 
\end{center}
\end{table}


\noindent In summary, Grady et al. were able to show that by implementing a population distributed genetic algorihtm optimal solution for scenario (a) can be obtained. They also showed that the power production obtained in Mosetti et al. could be increased by optimizing parameter values and using a more sophisticated implementation of the genetic algorithm. However, they make no attempt to compare their solutions for scenario (b) and (c) to solutions obtained using other optimization techniques. For a similar study where individuals are implemented as matrices in MATLAB see \citep{Emami}.\\


% Zhao et al.
\noindent \cite{Zhao} presented a very interesting study, where the electrical system of an off shore wind farm on Burko Bank in Liverpool Bay was optimized using a genetic algorithm. Although this is a study of cable clustering design, with fixed wind turbine count and positions, it is very interesting because it is compared against actual results obtained by the Burbo project team. To get an understanding of the optimization problem four example clustering designs are presented in figure \ref{Clustering Design}.\\


\begin{figure}[h!]
\begin{center}
\includegraphics[scale=0.2]{images/"Clustering Design (Zhao)"}
\caption{Four example clustering designs \citep{Zhao}.}
\label{Clustering Design}
\end{center}
\end{figure}


\noindent In their paper, Zhao et al. present an extensive study of different genetic algorithm techniques to find out which performs best on this type of optimization problem. Premature convergence is discovered as the main problem of the genetic algorithm and to deal with this different techniques are presented such as a diversity check, and a crowding technique called restricted tournament selection. For implementation details see reference \citep{Zhao}. Different genetic algorithm designs were tested, and the results show that the final design obtained was equal to the design obtain by the Burbo project team! This shows that optimization using sophisticated genetic algorithm implementations can find the same solution as current optimization techniques for optimization of electrical systems.\\


% Huang, 2007
\noindent \cite{Huang} presented a study showing that a population distributed genetic algorithm performs better than a simple genetic algorithm. Huang uses a more realistic objective function than the previous studies, taking into account the selling price of electric energy, as well as cost and energy production. The distributed genetic algorithm used the Island model, with 600 individuals divided among 20 demes, with the ring-topology shown in figure \ref{Ring Topology}.\\

\begin{figure}[h!]
\begin{center}
\includegraphics[scale=0.3]{images/"Ring Topology"}
\caption{Ring topology example with 6 demes \citep{Huang}.}
\label{Ring Topology}
\end{center}
\end{figure}


\noindent The simulation was run for 2500 generations with the migration strategy that 3.3\% of the individuals with highest fitness was selected as migrants, to replace the individuals with lowest fitness in the new population every 20th generation. The distributed algorithm was tested using the same three scenarios as \cite{Mosetti} and \cite{Grady}, and compared against a simple genetic algorithm. In case (a) the population distributed genetic algorithm was able to come up with the optimal solution (presented by Grady et al.), while as the simple genetic algorithm was not. For each of the three scenarios the population distributed algorithm ended up with higher fitness value, more power produced, lower CPU time and fewer generations. In case (a) the turbine count was equal for both algorithms, resulting in higher efficiency for the distributed algorithm. In scenario (b) and (c) the population distributed algorithm produced solutions with one more turbine than the simple algorithm, resulting in slightly lower efficiency. \\


% Mora
\noindent All studies presented above have used binary encoding in their wind farm representation, but \cite{Mora} presented a study where the binary encoding was replaced by an integer encoding. In their approach, an individual was represented as a set of $(x, y)$-coordinates representing turbine positions. In addition to optimize turbine position, Mora et al. also wanted to optimize both turbine type and height. In order to do this, their individuals was represented by a matrix where the first row contained the x-coordinates of the turbines, the second row the y-coordinates, the third row turbine type and the forth row turbine height as shown in figure \ref{Individual Representation (Mora)}. Note that with this type of encoding, different individuals can have different lengths, depending on the number of turbines in each solution.\\


\begin{figure}[h!]
\begin{center}
\includegraphics[scale=0.4]{images/"Individual Representation (Mora)"}
\caption{Representation of an individual of length $k$ (layout with $k$ turbines), where the first row represents x-coordinates of the turbines, the second row y-coordinates, the third row turbine type, and the forth row turbine height \citep{Mora}.}
\label{Individual Representation (Mora)}
\end{center}
\end{figure}


\noindent Five crossover methods, and a masked mutation method were presented for the new type of encoding, see \cite{Mora} for implementation details. To model the wind speed, the Weibull distribution was used, a more realistic wind speed model than the one used in the previous studies. The Weibull distribution will be explained in more detail in section \ref{subsection:wakewindpower}, since it is also used in the simulator for this thesis. Three different case studies are performed, the first one searches for an optimal solution when the number of turbines are decided beforehand. The second searches for an optimal positioning, type and height of turbines within a given budget. And the third one searched for an optimal solution with no such constraints. The results are only briefly discussed, however, this paper marks the shift from binary encoding to integer encoding, and from simple wind models to the Weibull distribution.\\


%Wan et al, 2009.
\noindent \cite{Wan} criticized the simple power-, and wind distribution model presented by references \citep{Mosetti, Grady}. Rather, they used the Weibull distribution as \citep{Mora} to model the wind, and they introduced a novel power model \\


\begin{equation}
\label{Power Model (Wan)}
\int_{u_{in}}^{u_{out}} P(u)f(u) du,
\end{equation}


\noindent where $u_{in}$ is the cut-in wind speed of the turbine, and $u_{out}$ is the cut-out wind speed of the turbine. $P(u)$ is the power output for the wind speed $u$ and $f(u)$ is the probability density of the wind speed $u$. The genetic algorithm was similar to that of Grady et al., and results show that the produced power increases.\\


% Kusiak et al., 2009
\noindent One of the most complete studies of the wind farm layout optimization problem was done by \cite{Kusiak}. The study is based on six assumptions, which according to the authors are realistic and industrial-accepted. The study assume a fixed, predetermined turbine count, small variations of surface roughness, turbines with equal power curves, wind speed following the Weibull distribution, that wind speed at different locations with same direction share the same Weibull distribution, and last, it assumes that any two turbines must be separated with at least four rotor diameters. A multi-objective function was used to calculate the fitness of the solutions. It consisted of one objective function to maximize expected energy produced, and one to minimize the constraint violations. Kusiak et al. critiques Mosetti et al. and Grady et al. for not basing their wind energy calculation on the power curve function and not thoroughly discussing wind direction. Their work include an extensive model of wind energy based on a discretization of the expected power production for each wind direction. Their algorithm was tested on real wind data, and compared against an upper bound on power production (power produced without wake effect), and their results show that less than 2\% of power is lost due to wake effects when 6 turbines are positioned in the wind farm. \\


% Gonzalez et al. 2010.
\noindent Assumptions such as cost models only dependent on the number of turbines are unrealistic, and needs to be removed in order to model the wind farm layout optimization problem in an realistic way.  \cite{Gonzalez} introduced a cost model based on the net present value, which takes into account wind speed, wind distribution, the number-, type-, rated power- and tower height of turbines, loss to due wake effects, auxiliary costs, road infrastructure, buildings, substation, electrical framework and financial aspects such as return on investment. For example, In order to more accurately model civil cost they present an greedy search which connects wind turbines to auxiliary roads or other turbines dependent on their position. Figure \ref{Civil Cost Greedy Algorithm (Gonzales)} shows how the greedy algorithm works. First, the distance between each turbine and every road is calculated. Since turbine A is closest to road 1, they are connected. Second, the distance between the remaining turbines and the roads and turbines already connected are compared, and turbine C is connected to road 2. At last, turbine B needs to be connected. Since it is closer to turbine A, than turbine C or any of the roads, it is connected to turbine A.\\


\begin{figure}[h!]
\begin{center}
\includegraphics[scale=0.3]{images/"Civil Cost Greedy Algorithm (Gonzales)"}
\caption{Greedy algorithm to estimate civil cost. \citep{Gonzalez}.}
\label{Civil Cost Greedy Algorithm (Gonzales)}
\end{center}
\end{figure}


\noindent Other new features includes a local search used when the genetic algorithm cannot find a better individual, and a genetic algorithm that can manage forbidden areas and that gives penalties for turbines positioned in undesirable terrain. Individuals are represented the same way as in reference \citep{Mora} displayed in figure \ref{Individual Representation (Mora)}. Results are compared against Grady, and shows higher produced energy. The authors also include three case studies showing how their algorithm can handle restrictions such as roads crossing, forbidden zones, undesirable zones and maximum investment cost.\\


% Sisbot et al. 2010
\noindent \cite{Sisbot} published a case study of wind turbine placement on a wind farm at the Island G{\"o}k\c{c}eada, at the north east of the Aegean Sea. A distributed genetic algorithm was used, but unlike Huang, the individuals were evaluated based on multiple objective functions; one that measures the total cost (installation and operational), and one that measure total power production. \c{S}i\c{s}bot et al. argue that in an environment with changing demands, the use of a multi-objective function gives the decision-makers the opportunity to evaluate the different designs based on cost and power production separately, without ill-informed, randomly generated weights. The selection process used is a controlled, elitist process, meaning that not only the fittest, but also some individuals that can spread diversity to the population are selected for reproduction. The genetic algorithm returns a set of Pareto optimal solutions; a set of solutions that are not dominated by any other solution in the set. Stated more formally, solution \textbf{y} is said to dominate solution \textbf{x} if

\begin{equation}
\forall i: f_i(\textbf{x}) \leq f_i(\textbf{y}) \hspace{1mm} and \hspace{1mm} \exists j: f_j(\textbf{x}) < f_j(\textbf{y}) 
\label{Dominated solutions}
\end{equation}

where $f_i$ is objective function \citep{Murata}. Other interesting features of this study is the introduction of constraints on wind turbine positions and constraints on the cost, meaning that individuals with wind turbines outside the area of the island, and individuals with costs larger than the budget are removed from the population. Even though constraints on individuals are not in accordance with the nature genetic algorithms, they can be necessary when the algorithm is applied to a real problem. Another feature introduced in this paper is rectangular cells. The argument behind this decision is that the safe distance between wind turbines is dependent on the direction of the turbine. The minimum distance between turbines in prevailing wind is 8D, while the minimum distance between turbines in the crosswind is 2D. In spite of this attempt to make the wind scenario more realistic, it is critiqued because it operates with a constant wind direction and constant speed, using the average wind direction and speed measured at the Island \citep{Samorani}. Results are not compared against previous studies, and the argument behind this decision is that it is hard to compare a Pareto-optimal set of solution to one of the previous solutions. \\


% Saavedra-Moreno et al., 2010
\noindent Another very interesting solution to the wind farm layout optimization problem was proposed by \cite{Saavedra-Morena}. Four novel improvements were included in their model. First, a shape model was introduced to model the terrain shape. By introducing this model, the simplification of a square grid was lost, and every terrain shape could be implemented. Second, an orography model was used to model the wind speed on different heights. Using this technique, the wind model is much more realistic because it takes into account that wind speed differs at different heights. Third, they introduce a new cost model, which takes into account installation cost, connection between turbines, road construction and net benefit from the produced energy. The fourth, and maybe most exiting improvement presented  was that a greedy heuristic was used to decide the initial positioning of the turbines for some of the individuals. This improvement was even requested by Mosetti et al. in 1994, but not included by anyone until now. The greedy heuristic works by placing turbines one by one in the position with maximal wind speed. First, the first turbine is positioned in the position wind maximum wind speed, next the wind speed is updated because of the reduction in wind speed caused by the wake effect of the first turbine, third, the second turbine is positioned in the position with maximal wind speed. This process continues until $N$ wind turbines have been placed. Clearly, the resulting layout is largely influenced by the positioning of the first turbine, and leads to a sub-optimal solution on its own. However, it is much better than a random solution, and as it turns out a good starting point for the genetic algorithm. Results for 15 different orographys for the same terrain shape is provided, showing the objective function values obtained by the greedy heuristic, a simple genetic algorithm with random starting positions, and the seeded genetic algorithm (the genetic algorithm with starting positions provided by the greedy heuristic). In each case, the genetic algorithm with random starting positions beats the results obtained by the greedy heuristic, but more importantly, in each case, the seeded genetic algorithm beats the results of the simple genetic algorithm. \\


% Chen et al. 2013
\noindent Both \cite{Mora} and \cite{Gonzalez} used the genetic algorithm to optimize the height of the turbines, as well as other parameters by representing individuals as shown in \ref{Individual Representation (Mora)}. Another approach to optimize turbine height was presented by \cite{Chen}. They state that normally, the same turbine type can be bought with several different heights, and that it therefore makes sense to use different height turbines. To optimize turbine position and height, they used two nested genetic algorithms. The first one was used to optimize turbine positioning, while the second one was used to decide between two turbine heights. For each generation of the first genetic algorithm, the second one was run for 50 generations to optimized turbine height. Binary encoding was used for individual representation in both genetic algorithms as shown in figure \ref{Representation (Chen)}. The first binary string represent turbine positioning in the environment, while the second binary string represents turbine height for each position that contains a turbine. Several case studies were performed in the paper, and results show that turbine layout with different turbine height, produce more energy than same-height turbines every time. \\


\begin{figure}[h!]
\begin{center}
\includegraphics[scale=0.3]{images/"Representation (Chen)"}
\caption{Representation for both genetic algorithms from \citep{Chen}. (a) Binary string representing turbine positions, (b) binary string representing height of the given turbines positioned by the first genetic algorithm.}
\label{Representation (Chen)}
\end{center}
\end{figure}


% Gao et al.
\noindent \cite{Gao} also implemented a population distributed genetic algorithm to solve the wind farm layout optimization problem. Unfortunately, the implementation details have not been published. Integer encoding was used to represent individuals, but unlike those presented earlier \citep{Mora, Kusiak, Gonzalez, Saavedra-Morena}, each individual have the same number of turbines. The algorithm is tested on the same three scenarios from Mosetti et al. and compared against other studies \citep{Mosetti, Grady, Gonzalez, Wan}. For the comparison to be valid they force their solutions to use the same number of turbines as obtained in the results of the previous studies. In each case their resulting layout is able to produce more energy, and has higher efficiency, however, it never achieves the highest fitness. In addition to this comparison, Gao et al. introduce an interesting hypothetical case study of wind turbine placement on an offshore farm located in the Hong Kong southeastern water. By using real wind data, collected over 20 years, they demonstrate that the distributed genetic algorithm can be applied to a real-world wind farm layout optimization problem. The resulting wind farm layout was estimated to be able to produce 9.1\% of yearly electrical consumption in Hong Kong (2012).


\section{Wind Farm Layout Optimization Different Approaches}\label{section:relatedworkother}


%Ozturk et al., (2004)
A greedy heuristic approach to the wind farm layout optimization problem was presented by \cite{Ozturk}. The algorithm starts out with an initial solution, where a number of wind turbines are positioned in the wind farm. Next, the greedy algorithm tries to improve the layout by performing either an add operation, a remove operation or a move operations. The add operation works by randomly position one new turbine in the terrain a number of times at different locations, and observe the change in the objective function value. The remove operation works by observing the change in objective function value when a turbine is removed, the process is repeated for all turbines. A move operation consist of moving each turbine 4 rotor diameters away from its current position in eight wind direction on at a time, and observe the change in the objective function value. The operation actually performed by the algorithm is the one that improves the objective function value the most. The greedy heuristic often converged to a local optimum, and the authors try to cope with this problem by performing randomly perturbations on a number of turbine positions if no improvements could be found using the add-, remove-, or move operation. Three approaches were investigated to find the initial position of the turbines; (1) randomly positioning, (2) packing the wind farm with as many turbines as possible, and (3) start with zero turbines. Preliminary testing showed that the second approach produced best results. The greedy algorithm was tested, and the results were compared to a feasible solution with the maximum number of turbines positioned, i.e. the initial layout before the algorithm was run. In 10 out of 12 case studies the algorithm improved the layout of the wind farm, with an average improvement of 4.3\%. \\


% Bilbao 2009
\noindent \cite{Bilbao} designed a simulated annealing algorithm to solve the wind farm layout optimization problem. The same wind parameters and representation as \cite{Grady} was used, and the algorithm was tested on the same three scenarios. The simulated annealing algorithm works as follows; first, an initial layout is obtained by randomly positioning a predefined number of turbines. Later, a random position that contains a turbine is chosen, and a new, randomly generated location is suggested. If the new position is better, the turbine is moved, but if the new position is not better, the turbine is moved with a certain probability which is regulated by a decreasing temperature parameter, in order to prevent the algorithm of converging to a local optimum. In case (a) from \citep{Grady} the simulated annealing algorithm is able to find the same optimal solution, and that by using only $\approx 4\%$ of the execution time of Grady et al., and only $\approx 1\%$ of the time spent evaluating the solution. In case (b) and (c) the simulated annealing algorithm is able to find solutions with better fitness, higher power production, higher efficiency, and significantly lower execution- and evaluation times, showing that simulation annealing might be a good technique to search for the optimal wind farm layout, and it should definitively be tested in a more realistic environment. \\ 


% Eroglu 2012
\noindent \cite{Eroglu} proposed an ant colony algorithm for the wind farm layout optimization problem; an algorithm that is inspired by how ants search for food, and show other ants food sources based on leaving a pheromone trail. The algorithm operates on a predetermined number of turbines, randomly positioned. The pheromone quantity of each turbine is decided by calculating the wake loss for the given turbine, resulting in a stronger pheromone trail for turbines with worse locations. Ants will follow the pheromone trail, therefore more ants will try to better the position of the worse turbines by moving them in random directions - the turbine is only moved if the new position is better than the current. Results are compared against \citep{Kusiak} and it is shown that the ant colony algorithm was able to position two more turbines; eight turbines in total, and that when the number of turbines is greater than two, the current algorithm produce more power, has less wake loss and higher efficiency.\\


% Wan et al., 2012
\noindent Another technique to search for the best wind farm layout is swarm optimization, and \cite{Wan2012} demonstrates how a Gaussian particle swarm algorithm can solve the wind farm layout optimization problem. Swarm optimization, is an optimization technique inspired by fish schooling, insect swarms and bird flocking. The algorithm presented used an objective function that tries to maximize produced power, while minimizing constraint violations. The algorithm works as follows: First, $N$ particles are placed in random (x, y)-positions. Second, the initial solution is evaluated and the results are saved. Third, the population best position $z^g$ is saved, along with the current best position observed for each particle $Z^p$, which in the beginning will be the initial positions. An algorithm is presented, to decide which, out of two layouts, is the best. It works by first prioritizing layouts which violates less constraints, and second compare power produced by the two layouts. Forth, an updating scheme is run for a given number of iterations. It first checks if the local best position observed by that particle $z^p$ is equal to the global best position $z^g$, and if so, uses an regeneration scheme that moves the particle to a random position. Otherwise, a new position is calculated for the particle based on the particles current position, its current best observed position and the global best observed position and two normalized random Gaussian numbers. If the new position is better than the previous one, the particle is moved. A differential evolution local scheme is also incorporated in each iteration to improve the algorithms local search ability. It basically work by randomly picking three random particles as potential parents for a given particle, and combine these to generate an alternative new position, which is assign to the particle, if it is better than the current one. Results were compared against \citep{Grady}, and show that the power generated is higher using this algorithm. Their algorithm is also tested in a more realistic environment and compared against an empirical method as well as a simpler particle swarm algorithm, and shows that the power generated is increased using the proposed algorithm.\\


\section{Discussion Related Work}\label{section:relatedworkdiscussion}
In order to provide an overview of the different publications presented in the two previous sections the wake-, wind-, power-, and cost model, along with objective functions, type of genetic algorithms and a short explanation of novelties presented in each article is presented in table \ref{table:overview related work}. \\

\noindent As can be seen in the table, each article utilized some variation of the Jensen model, developed by \cite{Jensen}, and later improved upon by \cite{Katic} and \cite{Frandsen}, to model wind speed decay as a consequence of the wake effect. The same model will also be used in this thesis. \\

\noindent Even though not many improvements has been made to the wake model, the wind model has evolved a lot since \cite{Mosetti}. As presented before, the three wind scenarios developed by Mosetti et al.; (a) single wind direction, uniform intensity, (b) multiple wind direction, uniform intensity, and (c) multiple wind directions and intensity, are not very realistic. The Weibull distribution, introduced by \cite{Mora}, models the wind distribution way better, and has been adopted by everyone, except those who still wanted to compare their results against \cite{Mosetti} and \cite{Grady}. As mentioned before, the Weibull distribution will also be used to model wind distribution in this thesis, and will be described in more detail in section \ref{subsection:wakewindpower}. \\

\noindent The majority of the publications presented uses the power model presented by \cite{Mosetti}, however, \cite{Kusiak} present a more realistic, linear power model, which also will be used in this thesis and are explained in section \ref{subsection:wakewindpower}.\\ 

\noindent The quality of the cost model has varied greatly in the different studies. The very unrealistic cost model that only takes turbine count into account has been adopted by many, as can be seen in table \ref{table:overview related work}. However, \citep{Mora}, \citep{Gonzalez}, \citep{Sisbot}, \citep{Saavedra-Morena}, and \citep{Chen} used more realist cost models, taking into account different parameters such as net present value, installation cost, maintenance work, civil work, interest rate and so on. In the current thesis, a very complicated objective function is presented, one which takes into account turbine cost, substation cost, interest rate, operating costs, and turbine count as well as produced power. Section \ref{subsection:fitnessfunction} gives an detailed explanation of the objective function. \\

\noindent Methods other than the genetic algorithm also shows promising results in solving the wind farm layout optimization problem. Simulated annealing \citep{Bilbao}, ant-colony optimization \citep{Eroglu} and swarm optimization \citep{Wan2012} are popular algorithms within the artificial intelligence community, that should be optimized in order to solve the wind farm layout optimization problems. The greedy heuristic approach presented in \citep{Ozturk} need to be tested in a more realistic environments in order to find out if this method really could be used on turbine layout positioning.\\

\noindent In this thesis, the wake-, wind-, power-, and cost model, and objective function was provided by GECCO 2015, and therefore no attempt will be made in improving any of these. Since the simple genetic algorithm presented by \citep{Mosetti}, different approaches has been tested in order to improve the results. Already in 2005, it was shown that by using a population distributed genetic algorithm \citep{Grady}, and change a few parameters, the results from \citep{Mosetti} was improved. In \citep{Huang} the focus was on showing how population distributed genetic algorithms performs better than simple genetic algorithms on solving the wind farm layout optimization problem, and his results show that the population distributed genetic algorithm is never beaten by the simple genetic algorithm. Both Grady et al., and Huang et al., uses the Island model when implementing distributed genetic algorithms. Even though these results indicate that population distributed genetic algorithms works better, the results of Grady et al., is beaten by the simple genetic algorithm of \cite{Gonzalez} when a local search is used together with the genetic algorithm. Also, in \citep{Saavedra-Morena} it is shown that a seeded simple genetic algorithm shows promising results, even though it is not compared against a population distributed genetic algorithm. \cite{Gao} also demonstrates how their population distributed genetic algorithm performs better than \citep{Mosetti, Grady, Gonzalez, Wan}, but sadly they do not share many implementation details. These results clearly show that population distributed genetic algorithms can be an effective optimization technique for the wind farm layout optimization problem, and they are the motivation behind the goal statement and research questions. Together with the Island model, all the distributed algorithms presented in section \ref{subsection:dga} will be implemented and tested in this thesis.\\


\tabcolsep=0.1cm
\begin{sidewaystable}
\caption{\textcolor{red}{Caption!}}
\label{table:overview}
\tiny
%\footnotesize
\begin{center}
\begin{tabular}{l | l | l | l | l | l | l | l | l }
Publication                     & Wake Model       & Wind Model             & Power Model    & Cost Model  & Objective   & GA   & Representation & Novelties \\ 
\hline 
\cite{Mosetti}                & \cite{Jensen}   & Simple scenarios      & Betz                 & Simple        & Simple        & SGA	& Binary   & Novel. \\
\cite{Grady}                   & \cite{Jensen}   &	 Simple scenarios   &	Betz	             & Simple        &	 Simple     & DGA	& Binary   & Population distributed. \\
\cite{Huang}                  & \cite{Katic}      &	Simple scenarios    & Betz 	             & Simple        & Simple       & DGA	& Binary   & Compared DGA and SGA. Realistic objective. \\
\cite{Mora}                     & NA                   & Weibull distribution  & NA                   & Complex     & Complex    & SGA  & Integer  & Weibull distribution, integer encoding. \\
\cite{Emami}                  & \cite{Jensen}   & Simple scenarios      & Betz                & Simple        & Simple        & SGA   & Binary  & Matrix representation of individuals. \\
\cite{Wan}                     & \cite{Jensen}    & Simple scenarios      & Betz                & Simple        & Simple       & DGA   & Binary   & New power function. \\
\cite{Kusiak}                  & \cite{Jensen}    & Weibull distribution  & Linear              & NA             & LP             & SGA  & Integer   & Realistic environment. \\
\cite{Gonzalez}              & \cite{Frandsen} & Weibull                    & Betz                & Simple        & Complex    & SGA   & Integer  & Extensive cost model. \\
\cite{Sisbot}                  & \cite{Jensen}    & Simple scenarios      & Betz                & Simple        & LP             & DGA   & Binary   & Multi-objective with pareto ranking. \\
\cite{Saavedra-Morena} & \cite{Jensen}    & Weibull distribution  & Betz                & Complex     & Complex   & SGA    & Integer  & Seeded genetic algorithm. \\
\cite{Chen}                    & \cite{Frandsen} & Simple scenarios      & Betz                & Simple        & Complex   & SGA    & Binary   & Nested genetic algorithms. \\
\cite{Gao}                      & \cite{Jensen}    & Real wind data         & Betz                & Simple        & Complex    & DGA   & Integer  & Hong Kong case study.
\end{tabular} 
\end{center}
\end{sidewaystable}
