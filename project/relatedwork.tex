\chapter{Related Work}


\section{A Survey of the State of the Art}


% Mosetti et al, 1994.
In 1994, Mosetti et al. successfully utilized the genetic algorithm on the wind farm layout optimization problem {\citep{Mosetti}}. To model the wind decay Mosetti et al. used a model similar to the Jensen model \textcolor{red}{[Jensen, 1983]}. Mosetti et al. divided the wind farm terrain into a $10\times10$ quadratic grid so that a wind turbine could be installed in the middle of each cell. The optimization problem would then be to find which cells wind turbines should be installed in, in order to maximize power production and minimize cost. With this representation, an individual of the genetic search could be represented as a binary string of length 100, where each index represents a cell in the grid, so that a value of 1 means that an wind turbine is installed in the corresponding cell, and a value of zero means that there is no wind turbine in the corresponding cell. The genetic algorithm used was a simple, single-population genetic search where the fittest individuals where selected for reproduction using crossover and mutation. The crossover operation was performed at random locations with probability $0.6 < P_c < 0.9$ and mutation was performed with probability $0.01 < P_m < 0.1$. The fitness of the individuals was determined by the objective function

\begin{equation}
Objective = \frac{1}{P_{total}}w_1 + \frac{cost_{total}}{P_{total}}w_2
\label{Objective function Mosetti et al.}
\end{equation}

where $P_{total}$ is the total energy produced in one year, $cost_{total}$ is a function of the number of wind turbines installed, and $w_1$ and $w_2$ are weights. In the current study, $w_1$ was kept small so that the focus would be on lowest cost per energy produced. \\

\noindent The model was tested on a single type of turbines in three different scenarios (a) single wind direction, (b) multiple wind direction with constant intensity, and (c) multiple wind direction and intensity. For each scenario, the results was measured against random configurations of 50 turbines. In scenario one, the efficiency of the random configuration was 0.50, while the efficiency of the optimal solution was 0.95. In the second wind scenario, the efficiency was decreased from 0.35 in the random configuration to 0.88 in the optimized configuration. And, in the last scenario the efficiency was increased from 0.34 to 0.84. The genetic algorithm was able to reduce the total cost with approximately 50\% for each scenario, while the total energy production was decreased with 29\%. \textcolor{red}{Feil!} For each scenario the number of wind turbines was decreased drastically in the optimized version. Table \ref{Results Mosetti et al.} summarizes the results obtained.

\begin{center}
\begin{table}[h!]
\caption{Optimized configurations compared against random configurations for each of the three scenarios (a) single wind direction, (b) multiple wind direction with constant intensity and (c) multiple wind direction and intensity {\citep{Mosetti}}.}
\label{Results Mosetti et al.}
\scalebox{0.8}{
\begin{tabular}{c|c|c|c|c|c}
Scenario & Configuration & Efficiency & $P_{tot}$(kWyear) & cost/kWyear & Number of turbines \\ 
\hline 
(a) & Random    & 0.50 & 13025 & $2.57 \times 10^{-3}$ & 50 \\ 
      & Optimized & 0.95 & 12375 & $1.57 \times 10^{-3}$ & 25 \\ 
(b) & Random    & 0.35 & 9117   & $3.68 \times 10^{-3}$ & 50 \\ 
      & Optimized & 0.88 & 8711   & $1.84 \times 10^{-3}$ & 19 \\ 
(c) & Random    & 0.34 & 4767    & $7.04 \times 10^{-3}$ & 50 \\ 
      & Optimized & 0.84 & 3695    & $3.61 \times 10^{-3}$ & 15 \\ 
\end{tabular}}
\end{table}
\end{center}


As discussed in the paper, different simplifying assumptions are made in the model such as the wake effect model, the cost function, turbine type and layout model. The results are also only compared against random configurations, not configurations optimized by other optimization approaches, and no attempt is made to optimize the software. However, the purpose of this initial paper was to demonstrate the applicability of genetic algorithms on the wind farm layout optimization problem, and it laid the foundation for a number of studies of wind farm optimization using genetic algorithms. \\


% Grady et al. 2005.
\noindent In 2005, Grady et al. picked up where Mosetti et al. left of \textcolor{red}{[Grady et al., 2005]}. They recognized that while the results of Mosetti et al. beat random configurations they were not close to configurations made on simple empirical placement schemes. In their study, they opt to determine the effectiveness of the genetic algorithm compared to optimal configurations to investigate their usefulness in wind farm layout optimization. \\

\noindent As Mosetti et al., they also used a wake model similar to the Jensen model \textcolor{red}{(do I have to reference it again?)} as well as the same cost- and power function. However, the objective function was changed into

\begin{equation}
Objective = \frac{cost}{P_{tot}}
\label{Objective function Grady et al.}
\end{equation}

to minimize the cost per unit of produced energy, losing the weight from Mosetti et al presented in equation \ref{Objective function Mosetti et al.}.\\

\noindent The same three scenarios as Mosetti et al. was considered. However, the number of individuals was increased from 200 to 600, and run for 3000 generations instead of 400. Grady et al. also implemented a distributed genetic algorithm where the individuals was divided into 20 sub-populations, sadly not sharing any more implementations details. On the first scenario, Grady et al. recognized that with uniform wind distribution the optimal solution could be obtained by optimizing on single row of the layout, and copy it to the rest. As opposed to Mosetti et al., their results are identical to the optimal solution. In scenario (b) and (c) however, the optimal solution can not be obtained empirical, and therefore the results are just compared against those of Mosetti et al. The results for each scenario is displayed in table \ref{Results Grady et al.}. 


Table \ref{Results Grady et al.} compare the solutions of the two studies. The first thing to notice is the difference in number of turbines, where Grady et al. ends up with more turbines in each case, approximately doubling the number of turbines in scenario (b) and (c). The explanation behind this observation is the objective functions. Objective function \ref{Objective function Mosetti et al.} prioritizes low cost and hence prioritizes a lower turbine count. \textcolor{red}{Fact check!} The number of turbines for each case explains the largely explains the results. For each scenario the fitness of Mosetti et al. is higher than the fitness obtained in Grady et al. With exception of the first scenario, the efficiency is also larger in Mosetti et al., which makes sense since fewer turbines leads to less wake effect to decrease efficiency \textcolor{red}{Fact check!}. However, in each case, the total power production is largely increased in the current study, which also makes sense based on the turbine count \textcolor{red}{Fact check!}. 


\begin{table}[h!]
\begin{center}
\caption{Current results compared against the restults from Grady et al. for each of the three scenarios. \textcolor{red}{[Grady et al., 2005]}}
\label{Results Grady et al.}
\begin{tabular}{c|l|l|l}
Scenario & Parameter                    & Mosetti et al.  & Grady et al. \\ 
\hline 
(a)          & Fitness                        & 0.0016197     & 0.0015436 \\ 
              & Total power (kW year) & 12 352           & 14 310 \\ 
              & Efficiency (\%)            & 91.645            & 92.015 \\ 
              & Number of turbines     & 26                    & 30  \\ 
&&&\\
(b)         & Fitness                        & 0.0017371      & 0.0015666 \\ 
              & Total power (kW year) & 9244.7           & 17220 \\ 
              & Efficiency (\%)            & 93.859           & 85.174 \\ 
              & Number of turbines     & 19                   & 39 \\ 
&&&\\
(c)         & Fitness                        & 0.00099405   & 0.00080314 \\ 
              & Total power (kW year) & 13 460          & 32 038 \\ 
              & Efficiency (\%)            & 94.62            & 86.619 \\ 
              & Number of turbines     & 15                  & 39 \\  
\end{tabular} 
\end{center}
\end{table}


\noindent In summary, Grady et al. is able to show that an optimal solution for scenario (a) can be obtained by genetic algorithm search. It also shows that the power production obtained in Mosetti et al. can be increased by optimizing of parameter values and more sophisticated implementation of the genetic algorithm. However, they make no attempt to compare their solutions for scenario (b) and (c) to solutions obtained using other optimization techniques. \\


% Huang
\noindent Huang presented a study in 2007, showing that a distributed genetic algorithm performs better than a simple genetic algorithm \textcolor{red}{[Huang, 2007]}. Huang uses a more realistic objective function than the previous studies, taking into account the selling price of electric energy, as well as cost and energy production. The distributed genetic algorithm uses the Island model, with 600 individuals divided among 20 demes, using the ring-topology shown in figure \ref{Ring Topology}.

\begin{figure}[h!]
\begin{center}
\includegraphics[scale=0.3]{images/"Ring Topology"}
\caption{Ring topology example with 6 demes \textcolor{red}{[Huang, 2007]}.}
\label{Ring Topology}
\end{center}
\end{figure}


The simulation was run for 2500 generations with the migration strategy that 3.3\% of the individuals with highest fitness was selected as migrants, to replace the individuals with lowest fitness in the new population every 20th generation. The distributed algorithm was tested using the same three scenarios as Mosetti et al. and Grady et al., against a simple genetic algorithm. In case (a) the distributed genetic algorithm was able to come up with the optimal solution (presented by Grady et al.), while as the simple genetic algorithm was not. For each of the three scenarios the distributed algorithm ended up with higher fitness value, more power produced, lower CPU time and fewer generations. In case (a) the turbine count was equal resulting in higher efficiency for the distributed algorithm, while in scenario (b) and (c) the distributed algorithm produced solutions with one more turbine than the simple algorithm, resulting in slightly lower efficiency. Huang also briefly gives a thin explanation that the results obtained are slightly better than those of Grady et al., but acknowledges that the results are hard to compare because of different objective functions.\\


% Sisbot et al.
\noindent \c{S}i\c{s}bot et al. published a case study of wind turbine placement on a wind farm at the Island G{\"o}k\c{c}eada, at the north east of the Aegean Sea \textcolor{red}{[\c{S}i\c{s}bot et al., 2010]}. A distributed genetic algorithm was used, but, unlike Huang, the individuals were evaluated based on multiple objective functions; one that measures the total cost (installation and operational), and one that measure total power production. \c{S}i\c{s}bot et al. argue that in an environment with changing demands, the use of a multi-objective function gives the decision-makers the opportunity to evaluate the different designs based on cost and power production separately, without ill-informed, randomly generated weights. The selection process used is a controlled, elitist process, meaning that not only the fittest, but also some individuals that can spread diversity to the population are selected for reproduction. The genetic algorithm returns a set of Pareto optimal solutions; a set of solutions that are not dominated by any other solution in the set. Stated more formally, solutions \textbf{y} is said to dominate solution \textbf{x} if

\begin{equation}
\forall i: f_i(\textbf{x}) \leq f_i(\textbf{y}) \hspace{1mm} and \hspace{1mm} \exists j: f_j(\textbf{x}) < f_j(\textbf{y}) 
\label{Dominated solutions}
\end{equation}

where $f_i$ is objective function \textcolor{red}{[Murata et al., 2001]}. Other interesting features of this study is the introductions of constraints on wind turbine positions and constraints on the cost, meaning that individuals with wind turbines outside the area of the island, and individuals with costs larger than the budget are removed from the population. Even thought constraints on individuals are not in accordance with the nature genetic algorithms, they can be necessary when the algorithm is applied to a real problem. Another feature introduced in this paper is rectangular cells. The argument behind this decision is that the safe distance between wind turbines is dependent of the direction of the turbine. The minimum distance between turbines in prevailing wind is 8D, while the minimum distance between turbines in the crosswind is 2D. In spite of this attempt to make the wind scenario more realistic, it is critiqued because it operates with a constant wind direction and constant speed, using the average wind direction and speed measured at the Island \textcolor{red}{[Samorani, 2013]}. Results are not compared against previous studies, and the argument behind this decision is that it is hard to compare a Pareto-optimal set of solution to one of the previous solutions. \\


% Gonzalez et al. 2010.
\noindent Certainly wind farm layout optimization using genetic search needed to be tested in a more realistic environment, and in 2010 Gonzalez et al. successfully attempted to do just that \textcolor{red}{[González et al., 2010]}. Three main improvements were presented

\begin{itemize}
\item[1] A more realistic cost model that takes into account wind speed- and direction distributions, turbine count, type, rated power, and tower height, wake effect and other costs such as foundations, road work, building, electrical networks, maintenance and removal costs.
\item[2] A more realistic wake effect model based on \textcolor{red}{[Frandsen, 2006, 2007]}.
\item[3] An genetic algorithm that handles forbidden areas, soil capacity, roughness length for wind directions, and that are able to set a limit on the number of turbines or initial cost.
\end{itemize}

\noindent The genetic algorithm proposed was different than the previous ones on two points. First, for each new generation the turbine height- and type was optimized to find the maximum net present value (NPV). Second, if the best fitness was unchanged from one generation to another, a local search was performed to improve the fitness of the individuals even more if possible. \\

\noindent Results was compared against Grady et al., and the proposed algorithm was also able to find the optimal solution for wind scenario (a). For wind scenario (b), and (c), the proposed solution produced slightly more energy than that of Grady et al., and it needed a lot fewer generations to do so - which one would expect when local search is used.\\


% Gao et al.
\noindent In 2015, Gao et al. used a distributed genetic algorithm to solve the wind farm optimization problem \textcolor{red}{[Gao et al., 2015]}. Unfortunately, they have kept most the implementation details of the distributed genetic algorithm to themselves, but they introduce an interesting hypothetical case study of wind turbine placement on an offshore farm located in the Hong Kong southeastern water. By using real wind data, collected over 19 years, they demonstrates that the distributed genetic algorithm can be applied to a real-world wind farm layout optimization problem. It should also be noted that Gao et al. do not use the constraint that the turbines have to be positioned in the middle of a cell. This property is made possible by inspecting the positions of the turbines relative to each other, and reassign the turbines that are too close to other positions \textcolor{red}{They borrowed this method from Wan et al., 2009}.\\






